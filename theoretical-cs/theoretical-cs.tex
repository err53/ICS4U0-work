\documentclass[jou]{apa7}

\usepackage[
backend=biber,
style=apa,
citestyle=apa
]{biblatex}

\addbibresource{theoretical-cs.bib}

\title{Concurrency Theory and Applications}
\shorttitle{Concurrency Theory}
\leftheader{Huang}
\author{Jason Huang}
\authorsaffiliations{John Fraser Secondary School}
\course{ICS4U0}
\professor{Mr. Seidel}
\duedate{December 16, 2020}
\date{\today}

\abstract{
  Concurrency theory sounds like an abstract concept that would have no bearing on anything used in the real world.
  On the contrary, the field has led to the vast computing power seen in multi-core processors today,
  and the data integrity guarantees of modern storage systems.
  This paper will explain some of the practical applications of concurrency theory in use today.
}

\begin{document}

\maketitle

\section{Introduction}
Before we are able to define what concurrency theory is,
we must first define what concurrency is.

Concurrency is being able to overlap multiple running processes at the same time.
This does not require multiple cores,
but it does require a different approach to programming:
since it is not guaranteed that a set of instructions you write will all be processed at around the same time,
you need to take into account other processes that may access the same resource you are accessing.

For example, if two concurrent processes are incrementing a variable,
each process needs to perform 3 steps:

\begin{enumerate}
  \item Getting the value from the variable
  \item Incrementing the value
  \item Updating the variable with the new value
\end{enumerate}

Say that process A runs first, and gets the value from the variable.
A problem occurs if process B gets the variable after A has retrieved the variable, but before it writes the value back.
Instead of the value in the variable being incremented twice, it only gets incremented once.

A common way to accomplish this is through the use of mutual exclusion.
By providing some way for different processes to lock and unlock variables,
it can be ensured that data is not overwritten.
\autocite{lamportTuringLectureTheComputer2015}

\section{Collaborative Research Field}
The implementation of concurrency is incredibly important in many fields,
but this paper will only touch on one of them.

In the growing world of financial technology, or fintech,
data integrity is incredibly important.
Whether user data, financial records and transactions, or internal company data,
data loss is simply not an option when managing people's money.

Furthermore, the technology part of fintech is intended to make financial services more accessible to the general public.
However, more accessibility also means a larger user base,
and more actions being performed by users during the same time.
Billions of transactions happen daily,
and the software that runs these services must be able to keep up with the demand.
\autocite{foggSoftwareTestingFintech2020}

By taking into account the developments in the field of concurrency,
the field of fintech would be able to service more users,
while still keeping high data integrity guarantees.

\section{Careers}
A career that exists in the collaborative research field discussed above would be a financial analyst.
In a financial firm, analysts help the firm make investment decisions,
and in corporate firms,
financial analysts predict how the company's products will perform in the future,
and where the company should allocate more resources to improve profits.
This is incredibly important when running a finance company,
especially if the company would for part of their profits to come from investing deposited or managed money.
\autocite{segalBecomingFinancialAnalyst2020}

Another career that directly relates to this research field are database analysts, or DBAs.
DBAs are in charge of designing, deploying, and administering databases.
They are crucial for managing large amounts of data,
while keeping said data quickly accessible to users.
Database analysts work directly with developers and the IT department to plan new database deployments,
implement said deployments,
and making sure the databases are secure and performing well.
\autocite{jobheroDatabaseAnalystJob2020}

\section{Education}
Becoming a database analyst generally requires at least a bachelor's degree in either computer science or mathematics,
as well as having programming experience.
\autocite{governmentofcanadaDatabaseAnalystCanada2020}
Specifically, many jobs would like knowledge in languages like
SQL, Javascript, CSS, HTML, and PHP, as well as database and data modelling experience.
\autocite{jobheroDatabaseAnalystJob2020}
There are also a fair few certifications that may prove to be useful,
such as the MCSE: Data Management and Analytics certification,
designed for more senior DBAs,
which requires a database proficiency exam,
as well as a prerequisite MCSA certification in databases,
business intelligence,
or machine learning
\autocite{aroraBestDataAnalytics2020}

\printbibliography

\end{document}
